\section{Accented Characters}

Accent marks are foreign to the English alphabet, and mainly occur in loanwords,
where they are usually optional.

\subsection{Acute Accent (´)}

The acute accent mark is the most frequent in English. It is usually found in
\gr{é} and most usually marks a final \gr{e} in a French (or occasionally
Spanish) word which is pronounced \ph{eI} and does not cause magic \gr{e}
(\eg{café, passé, sauté, soufflé}). Serving the same function, it may also
appear in the combination \gr{ée} (\eg{divorcée}). In other contexts, it most
frequently serves no function and is equivalent to an unaccented character.

\subsection{Grave Accent (`)}

In the suffix \gr{ère} of French words, marks a pronunciation \ph{e@r};
otherwise equivalent to an unaccented character.

\subsection{Diaeresis (¨)}

Some English writers and editors use the diaeresis mark from Greek and French to
mark seperation of two vowels in English (\eg{coöperate, coördinate}). This is
especially frequent in the word \eg{naïve}.
