\section{Context-Sensitive Consonants}

\subsection{\gr{c} (\ph{s} and \ph{k})}

When followed by \gr{ae}, \gr{i}, \gr{e}, or \gr{y}, \gr{c} has the sound \ph{s}
(at the beginning of a word: \eg{Caesar, citrus, central, cyanide}; in the
middle: \eg{receipt, publicize, policy}). The digraph \gr{ce} at the end of a
word has the sound \ph{s}: it effects magic \gr{e} (\eg{pace, ice, reduce}),
except in the combination \gr{nce}, where it affects broad \gr{a} (\eg{mince,
nonce, whence, dance}).

\begin{exceptions}{\gr{ce} as \ph{s} not causing magic \gr{e}}
\item lettuce
\end{exceptions}

In \gr{cia} or \gr{cea} when not at the beginning of a word, \gr{c} has the
sound \ph{S} (\gr{social, ocean}).

In all other contexts, \gr{c} has the sound \ph{k} (\eg{cat, cod, cut, create,
logic}).

\begin{exceptions}{\gr{c} as \ph{k} before \gr{ae}, \gr{i}, \gr{e}, or \gr{y}}
\item Celtic
\end{exceptions}

\gr{cc} sounds as \ph{ks} in the contexts where the second \gr{c} would stand
for \ph{s} according to the rules above (\eg{accident, success}) and for \ph{k}
alone elsewhere (\eg{account, occupy}). It marks the previous vowel as short.

\begin{exceptions}{\gr{cc} as \ph{s} before \gr{ae}, \gr{i}, \gr{e}, or \gr{y}}
\item flaccid
\end{exceptions}

\subsection{\gr{ch} (\ph{tS} and \ph{k} (\ph{x}), \ph{S})}

There is very little regularity to the use of \gr{ch} to represent \ph{tS},
\ph{k}, or \ph{S} that is able to be explained without reference to word
etymology. The former is the more frequent value which may be considered the
standard realization. The second occurs mainly in words of Greek or Greco-Latin
origin as well as some more recent loanwords, but e.g. \eg{ache} is of Old
English origin. This situation is complicated further by the possibility of it
representing \ph{x}, in Scottish words (most familiarly \eg{loch}) and foreign
loans (\eg{Chanukah}), which may be rendered as \ph{k} by speakers who lack
\ph{x} in their vowel inventory. In \eg{arch} as a word it has \ph{tS}; in
\gr{arch} as a prefix (\eg{archangel, archaeology}) it has \ph{k}. The latter is
mainly in words of French origin (\eg{creche, cliché, machine}).

About the only failsafe rule that can be given for it is that before \gr{r} it
is always \ph{k}.

\subsection{\gr{gh}}

Before a vowel, \gr{gh} has the sound \ph{g} (\eg{Afghan, ghost, ghetto,
spaghetti}).

Elsewhere, \gr{gh} is silent. When it follows a single vowel, the vowel is
lengthened (\eg{Hugh, right}). When it follows \gr{ou} and precedes \gr{t}, and
when it follows \gr{au}, the sound becomes \ph{O:}. When it follows \gr{ou}
elsewhere, the sound is usually \ph{oU} (\eg{borough, dough}) (which is weakened
to \ph{@} in British English when it is not the first syllable).

\begin{exceptions}{\gr{ough} not pronounced \ph{u:}}
\item through
\item plough
\end{exceptions}

\subsection{\gr{ng} (\ph{N} and \ph{Ng}) and \gr{nge} (\ph{ndZ})}

Where the combination \gr{ng} occurs at the end of a morpheme, it indicates
\ph{N} (\eg{singing, clingy}). Within a single morpheme, it denotes \ph{Ng}
(\eg{linger, monger}).\footnote{In Northern England and some other areas,
\ph{N} is merged with \ph{Ng} except at the end of words; i.e. \eg{singer} and
\eg{linger} rhyme.}

In the combination \gr{nge} at the end of a morpheme, it sounds as \ph{ndZ} and
effects magic \gr{e} on the vowel \gr{a} but no others (\eg{change, range}; but
\eg{revenge, binge, plunge}). This sometimes applies in other environments
(\eg{angel, danger, dungeon}), sometimes with an exception to vowel lengthening
(\eg{tangent}).

\subsection{\gr{j} and \gr{dg} (\ph{dZ} and \ph{Z})}

Except at the end of words, \gr{j} has the sound \ph{dZ}. This is most common at
the start of morphemes (\eg{just}, \eg{jam}, \eg{juice}), though it does occur
in the middle of words, most often in Latin-derived terms and loanwords such as
foreign proper names (\eg{trajectory}, \eg{Fiji}).

At the end of morphemes and in the middle of words, especially where consonant
doubling is required to mark a preceding short vowel, \ph{dZ} is usually spelt
\gr{dg} (\gr{dge}). This digraph cancels the effect of any magic \gr{e}
following it.

\subsection{\gr{s} (\ph{s} and \ph{z})}

The letter \gr{s} alone between two vowels is voiced to \ph{z} except after
\gr{a}. This applies even when the following vowel is silent.

\begin{exceptions}{\gr{s} as \ph{z} between two vowels}
\item abuse (noun)
\item practise
\item precise
\end{exceptions}

\gr{ss} has the value \ph{s}.

\begin{exceptions}{\gr{ss} as \ph{z}}
\item dessert
\item dissolve
\item possess
\item scissors
\end{exceptions}

\gr{ssion} and \gr{ssian} make the syllable \ph{S@n} (\eg{commission, expression, passion, profession; Hessian, Russian}).

\subsection{\gr{t} and \gr{ti} (\ph{t}, \ph{S}, \ph{tS})}

The letter \gr{t} usually stands for the sound \ph{t} and behaves like a normal
consonant. However, when followed by \gr{i} it can also have the sound \ph{S} in
the suffixes \gr{tion} (\ph{S@n}) and \gr{tiate} (\ph{Si.eIt}) and its inflexions.

\gr{t} is also silent in the combination \gr{tch}, which sounds as \ph{tS} with
a short preceding vowel.
