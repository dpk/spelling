\section{The Diaphonemes of English}

English is a world language. The global diversity of pronunciation in English is
far greater than any other single language. Many pairs of words are homophonous
to a typical British speaker but not to a typical American. Even within British
English and American English, many variants exist. It is almost impossible to
make any general statement about any single specific aspect of 

Yet the same spelling system serves all these diverse dialects. Why should it
not? There is no dialect of English whose pronunciation is so at odds with the
mainstream that it cannot be understood.

For this reason, to talk about the \emph{phonemes} of English is misleading.
A set of phonemes implies a unity in pronunciation which does not really exist.

In the table below only those diaphonemes which are encoded, reliably or not, by
the spelling system of English are included. This specifically excludes, for
example, the \eg{bad}–\eg{lad} split of Australian English.

\subsection{Consonants}

\subsubsection{Stops}

\begin{tabular}{r l r l}
\ph{k} & \emph{k}ind, pi\emph{ck} & \ph{g} & \emph{g}uy, bi\emph{g} \\
\ph{p} & \emph{p}ie, s\emph{p}y, li\emph{p} & \ph{b} & \emph{b}it, cri\emph{b} \\
\ph{t} & \emph{t}oy, s\emph{t}ill, bi\emph{t} & \ph{d} & \emph{d}ip, hea\emph{d}
\end{tabular}

\subsubsection{Fricatives}

\begin{tabular}{r l r l}
\ph{f} & \emph{f}ind, i\emph{f} & \ph{v} & \emph{v}an, ha\emph{v}e \\
\ph{T} & \emph{th}ing, tru\emph{th} & \ph{D} & \emph{th}is, wi\emph{th} \\
\ph{s} & \emph{s}at, pa\emph{ss} & \ph{z} & \emph{z}one, wa\emph{s} \\
\ph{S} & \emph{sh}ine, ba\emph{sh} & \ph{Z} & bei\emph{g}e \\
(\ph{x} & lo\emph{ch}) \\
\end{tabular}

\begin{centering}
\subsubsection{Affricates}

\begin{tabular}{r l r l}
\ph{tS} & \emph{ch}ild, whi\emph{ch} & \ph{dZ} & \emph{j}ust, bri\emph{dg}e \\
\end{tabular}
\end{centering}

\subsubsection{Approximants}

\begin{tabular}{r l}
\ph{h} & \emph{h}igh \\
\ph{j} & \emph{y}es \\
\ph{l} & \emph{l}ead, hau\emph{l} \\
\ph{r} & \emph{r}ed, poo\emph{r}\footnote{Silent in some contexts in some accents.} \\
\ph{w} & \emph{w}ant \\
\ph{hw} & \emph{wh}at \\
\end{tabular}

\subsubsection{Nasals}

\begin{tabular}{r l}
\ph{m} & \emph{m}ake, ai\emph{m} \\
\ph{n} & \emph{n}ice, li\emph{n}e \\
\ph{N} & thi\emph{ng} \\
\end{tabular}

\subsection{Vowels}

\subsubsection{Simple Vowels}

\begin{tabular}{r l r l}
\ph{\ae} & tr\emph{a}p & \ph{A:} & sp\emph{a} (p\emph{a}lm) \\
\ph{I} & k\emph{i}t & \ph{i:} & fl\emph{ee}ce \\
\ph{U} & f\emph{oo}t & \ph{u:} & g\emph{oo}se \\
\ph{6} & l\emph{o}t & \ph{O:} & th\emph{ough}t \\
\ph{2} & str\emph{u}t \\
\end{tabular}

\subsubsection{Diphthongs}

\begin{tabular}{r l}
\ph{aI} & pr\emph{i}ce \\
\ph{aU} & m\emph{ou}th \\
\ph{eI} & f\emph{a}ce \\
\ph{oU} & g\emph{oa}t \\
\ph{OI} & ch\emph{oi}ce \\
\end{tabular}

\subsubsection{Rhotic Vowels}

\begin{tabular}{r l}
\ph{Ar} & st\emph{ar}t \\
\ph{U3r} & n\emph{ur}se \\
\ph{I3r} & f\emph{ir} \\
\ph{E3r} & l\emph{ear}n \\
\ph{e@r} & squ\emph{ar}e \\
\ph{i@r} & n\emph{ear} \\
\ph{o@r} & f\emph{or}ce \\
\ph{Or} & n\emph{or}th \\
\ph{u@r} & c\emph{ure} \\
\end{tabular}

\subsubsection{Weak Vowels}

\begin{tabular}{r l}
\ph{@} & comm\emph{a} \\
\ph{@r} & lett\emph{er} \\
\ph{i} & happ\emph{y} \\
\end{tabular}
