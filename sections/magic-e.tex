\section{Magic \gr{e}}

In the exceptions list below, only words whose \emph{only} spelling defect is an
erroneous \gr{e} which appears to, but which actually does not, lengthen its
preceding vowel. \eg{come}, for instance, would be incorrectly pronounced
\ph{koUm} if we are to trust its spelling, but without the \gr{e} it would be
the equally incorrect pronunciation \ph{k6m}.

\begin{exceptions}{Words where final \gr{e} is not magic}
\item have
\item give
\item opposite (compare \eg{termite, socialite}, etc.)
\item practise, practice
\item 
\item any Latin-derived term ending \gr{ive} (compare \gr{ize} and \gr{ise})
\end{exceptions}

The exceptions to the rule with unpredictable behaviour all have \gr{o} as their
root vowel, where a \gr{u} (with or without magic \gr{e}) would be expected:

\begin{exceptions}{Words where final \gr{e} has completely unpredictable behaviour, grouped by consonant and sound}
\item come, some (compare \eg{chrome, gnome, home, ,} etc.)
\item above, dove, glove, love, shove (compare \eg{drove, grove, Shrove, strove}, etc.)
\item improve, approve, move (see above)
\end{exceptions}

